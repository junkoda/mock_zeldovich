\documentclass[a4paper,11pt]{article}
\usepackage{amsmath}
\usepackage{amsthm}
\usepackage{amssymb}
%\usepackage{fancyhdr}
\usepackage{hyperref}

% Use Times fonts
%\usepackage{mathptmx}
\usepackage[scaled]{helvet}
\renewcommand{\ttdefault}{pcr}
\usepackage{bm}

\setlength{\oddsidemargin}{0pt}   %%% left margin
\setlength{\textwidth}{159.2mm}
\setlength{\topmargin}{0mm}
\setlength{\headheight}{10mm}
\setlength{\headsep}{10mm}         %%% length between header and test
\setlength{\textheight}{219.2mm}

\setlength{\parindent}{0pt}

%%
\newtheorem{formula}{Formula}[section]

%\pagestyle{fancy}
%\renewcommand{\headrulewidth}{0pt}
%\fancyhf{}
%\rhead{JK 0000 - \thepage}
%\rfoot{\url{http://www.github.com/junkoda/mock_zeldovich}}

\begin{document}

\section{Zeldovich approximation}


\section{Analytical Theory}

\begin{equation}
\begin{split}
  P_Z(k) &= \int \! d^3 r e^{-i \bm{k}\cdot\bm{r}}
          \left[ e^{-\frac{1}{2} \Sigma(\bm{k}, \bm{r})} - 1 \right]\\
       &= 4 \pi \int_0^1 \! d\cos\theta \int_0^\infty \! r^2 dr \,
          e^{-ikr\cos\theta} \left\{
          e^{k^2\left[ \frac{1}{3} \Psi_0(r) -
                       (\cos^2\theta - \frac{1}{3}) \Psi_2(r) - \sigma_v^2
                \right]}
            - e^{-k^2 \sigma_v^2} \right\}.
\end{split}
\end{equation}

The Zeldovich approximation is the first-order solution to the
Lagrangian pertubation theory, which relates the unperturbed
Lagrangian coordinate $\bm{q}$ to the pertubed Eularian position
$\bm{x}$,
\begin{equation}
  \bm{x} = \bm{q} + \bm{\Psi}(\bm{q})
\end{equation}

The displacement in redshift space is only difference by a constant factor,
\begin{equation}
  \bm{s} = \bm{x} + \frac{1}{aH(a)} v_z \bm{e}_z
         = \bm{q} + \bm{\Psi}^s(\bm{q}),
\end{equation}
where we assume that the line of sight is in the $z$ direction.

\begin{align}
  \begin{cases}
  \Psi_i^s &= \Psi_i \mbox{  for $i=1,2$}\\
  \Psi_3^s &= \Psi_3
  \end{cases}m
\end{align}

The Zeldovich appromation to the displacement vector $\Psi$ is,
\begin{equation}
  \delta(\bm{x}) = - \bm{\nabla}_q \cdot \bm{\Psi}(\bm{q})
\end{equation}
where $\delta$ is the density contrast $\delta = n(\bm{x}/\langle n
\rangle - 1$ for some density field n; the angle braket denotes the
ensamble average. The vorticity is a decaying mode and often set to zero,
\begin{equation}
  \bm{\nabla}_q \times \bm{\Psi}(\bm{q}) = 0.
\end{equation}

\begin{equation}
  \delta(\bm{k}) = \int d^3 q \, e^{-i \bm{k} \cdot \bm{q}}
                                 e^{-i \bm{k} \cdot \bm{\Psi}(\bm{q})}
\end{equation}

The power spectrum, defined by,
\begin{equation}
  \left\langle \delta(\bm{k}) \delta(\bm{k}') \right\rangle
  = (2\pi)^3 \delta_D(\bm{k} - \bm{k}') P(\bm{k}),
\end{equation}
is
\begin{equation}
  P(\bm{k}) = \int d^3 r \,  e^{-i \bm{k} \cdot \bm{r}} \left \langle
  e^{-i \bm{k} \cdot \left[ \bm{\Psi}(\bm{p}) - \bm{\Psi}(\bm{q}) \right]} \right \rangle,
\end{equation}
where $\bm{r} \equiv \bm{p} - \bm{q}$.

The ensamble average can be computed with the charactersic function formula
for a random Gaussian variable $X$ with $\langle X \rangle = 0$,
\begin{equation}
  \left\langle e^{-t X} \right\rangle = e^{-\frac{1}{2} \sigma^2 t^2},
\end{equation}
where $\sigma^2 = \left\langle X^2 \right\rangle$.

\begin{equation}
  P(\bm{k}) = \int d^3 r \, e^{-i \bm{k} \cdot \bm{r}}
    e^{-\frac{1}{2} \Sigma(\bm{k}, \bm{r})},
\end{equation}
where $\Sigma(\bm{k}, \bm{r})$ is the variance of $\bm{k} \cdot \left[
  \bm{\Psi}(\bm{p}) - \bm{\Psi}(q) \right]$:

\begin{equation}
  \Sigma(\bm{k}, \bm{r}) = 2 \sum_i k_i^2 \left[ \sigma_i^2 - \xi_{ii}(\bm{r}) \right],
\end{equation}
where
\begin{align}
  \sigma^2_i       &= \left\langle \Psi_i^2 \right\rangle,\\
  \xi_{ij}(\bm{r}) &= \left\langle \Psi_i(\bm{p}) \Psi_j(\bm{q}) \right\rangle
                   = \delta_{ij} \left\langle \Psi_i(\bm{p}) \Psi_j(\bm{q}) \right\rangle,
\end{align}
and $\delta_{ij}$ is the Kronecker delta.

The Zeldovich displacement field in Fourier space is,
\begin{equation}
  \bm{\Psi}(\bm{k}) = \frac{i\bm{k}}{k^2} \delta(\bm{k}).
\end{equation}



\begin{equation}
  \xi_{ij}(\bm{r}) = \int \frac{d^3 k}{(2\pi)^3}\,
                    e^{i\bm{k} \cdot \bm{r}} \frac{k_i}{k} \frac{k_j}{k}
                    \frac{P(\bm{k})}{k^2}
\end{equation}

The off-diagonal terms are zero from the asymmetry across the axies
and the diagonal term can be computed with Legendre polynomials,

\begin{align}
  \mathcal{P}_0(\mu) = 1\\
  \mathcal{P}_2(\mu) = \frac{1}{2} \left( 3 \mu^2 - 1 \right)
\end{align}


\begin{equation}
  \xi_{22}(\bm{r}) = \frac{1}{3} \int \frac{d^3 k}{(2\pi)^3} \,
             \left[ 2 \mathcal{P}_2(\mu) + \mathcal{P}_0 \right] P(\bm{k})/k^2
             = \frac{2}{3}  \xi_2(k) - \frac{1}{3} \xi_2(0)
\end{equation}
where $\xi_\ell$ is the multipoles of the displacement correlation function

\section{Discrete Theory}

\section{Formulae}

\begin{equation}
  \int_{-1}^1 P_\ell(\mu) P_m(\mu) d\mu = \frac{2}{2 \ell + 1} \delta_{\ell m}
\end{equation}

\begin{equation}
  4\pi \int_0^\infty r^2 dr j_\ell(kr) j_\ell(k'r) =
    \frac{2\pi^2}{k^2} \delta_D(k - k')
\end{equation}

\subsection{Numerical check}

Inverse spherical-Bessel transform:
\begin{align}
  P(k) &= 4 \pi k^2 \int \Psi_0(r) j_0(kr) r^2 dr\\
  P(k) &= 4 \pi k^3 \int \Psi_1(r) j_1(kr) r^3 dr
\end{align}

Linear limit:
\begin{equation}
  e^{-\frac{1}{2} \Sigma(k, r)} - 1 \approx -\frac{1}{2} \Sigma(k, r)
\end{equation}

\begin{equation}
  \begin{split}
  &\frac{1}{2} \int_{-1}^1 \! d\cos\theta \, e^{-ikr \cos\theta}
  \left\{
  \Psi_1(r) - \sigma_v^2 + \left[ \Psi_0(r) - 3 \Psi_1(r) \right] \cos^2 \theta
  \right\}\\
  &= \left[ \frac{1}{3}\Psi_0 - \sigma_v^2 \right] j_0(kr)
     + 2\left[ \Psi_1 - \frac{1}{3}\Psi_0 \right] j_2(kr)
  \end{split}
\end{equation}


\begin{equation}
  \cos^2 \theta = \frac{1}{3}\left[ P_0(\cos\theta) + 2 P_2(\cos\theta) \right]
\end{equation}
  
  
\begin{equation}
  P(k) = 4\pi k^2 \int r^2 dr \left\{
  \left[\frac{1}{3} \Psi_0(r) - \sigma_v^2 \right] j_0(kr)
  + 2 \left[ \Psi_1(r) - \frac{1}{3} \Psi_0 \right] j_2(kr)
  \right\}
\end{equation}

\begin{equation}
  \begin{split}
    \frac{j_1(z)}{z} &= \frac{\sin z}{z^3} - \frac{\cos z}{z^2} \\
    &= \frac{1}{3} j_2(z) + \frac{1}{3} j_0(kr)
  \end{split}
\end{equation}

\begin{equation}
  \langle \Psi_i(p) \Psi_j(q) \rangle
  = \Psi_\parallel(r) \hat{r}_i \hat{r}_j
  + \Psi_\perp(r) (\delta_{ij} - \hat{r}_i \hat{r}_j)
\end{equation}

\begin{align}
  \Psi_\parallel(r) &= \frac{1}{2\pi^2} \int \! dk \, P(k)
  \left[j_0(kr) - 2 \frac{j_1(kr)}{kr} \right] \equiv \Psi_0(r) - 2 \Psi_1(r)\\
  \Psi_\perp(r) &= \frac{1}{2\pi^2} \int \! dk \, P(k)
  \frac{j_1(kr)}{kr} \equiv \Psi_1(r)
\end{align}

\begin{align}
  \Psi_0(kr) &\equiv \frac{1}{2\pi^2} \int \! dk \, P(k) j_0(kr) \\
  \Psi_1(kr) &\equiv \frac{1}{2\pi^2} \int \! dk \, P(k) \frac{j_1(kr)}{kr}
\end{align}

\begin{align}
  \left\langle \Psi_i(\bm{p}) \Psi(\bm{q}) \right\rangle
  &= \int d^3 d_1 d^3 k_2 \, \frac{k_{1, i}}{k_1^2} \frac{k_{2,j}}{k_2^2}
     \langle \delta(\bm{k}_1) \delta(\bm{k}_2)^* \rangle e^{i\bm{k}_1 \cdot \bm{p}}
     e^{-i\bm{k}_2 \cdot \bm{q}}\\
  &= \int \! d^3 \hat{k}_i \hat{k}_j \frac{P(k)}{k^2} e^{i\bm{k}\cdot \bm{r}} 
\end{align}

\appendix

\section{Formulae}

\begin{formula}
  Plane wave expansion:
  \begin{equation}
    e^{i\bm{k} \cdot \bm{r}} = \sum_{\ell=0}^\infty (2 \ell + 1) i^\ell j_\ell(kr) \mathcal{P}_\ell (\cos \theta),
  \end{equation}
where $\bm{k} \cdot \bm{r} = k r \cos\theta$.
\end{formula}

$j_\ell$ is a spherical Bessel function,
\begin{align}
  j_0(z) &= \frac{\sin z}{z}\\
  j_2(z) &= \left( \frac{3}{z^3} - \frac{1}{z} \right) \sin z
            - \frac{3}{z^2} \cos z
\end{align}
and $\mathcal{P}_\ell$ is a Legendre polynomial,
\begin{align}
  \mathcal{P}_0(x) &= 1\\
  \mathcal{P}_2(x) &= \frac{1}{2} (3 x^2 - 1)
\end{align}

\begin{formula}
  Orthogonality of Legendre polynomials:
  \begin{equation}
    \int \mathcal{P}_m(\mu) \mathcal{P}_n(\mu) d\mu =
    \frac{2}{2 n + 1} \delta_{\ell n}.
  \end{equation}
\end{formula}

\begin{formula}
  Orthogonality of spherical Bessel functions:
  \begin{equation}
    \int \! r^2 dr j_\ell(kr) j_\ell(k'r)
    = \frac{\pi}{2k^2} \delta_D(k - k').
  \end{equation}
\end{formula}

\begin{formula}
  \label{formula:cos2}
  \begin{equation}
    \int d\Omega_k \cos^2 \theta e^{i\bm{k}\cdot\bm{r}}
    =  4\pi \left[ j_0(kr) - 2 j_2(kr) \right]
  \end{equation}
\end{formula}

\begin{proof}
  Legendre polynomials $P_0 = 1$ and $P_2(\cos\theta) = \frac{1}{2} (3 \cos^2\theta - 1)$ give,
  \begin{equation}
    \cos^2 \theta = \frac{1}{3}( P_0 + 2P_2 )
  \end{equation}
  The plane-wave expation and the orthognality of Legendre polynomials give the formula.
\end{proof}

%
% k_i k_j exp(ikr)
%
\begin{formula}
  \label{formula-moment}
  \begin{equation}
    \frac{1}{4\pi}\int d\Omega_k \, \hat{k}_i \hat{k}_j e^{i\bm{k}\cdot\bm{r}}
    = \frac{1}{3} \delta_{ij} j_0(kr) - \left(\hat{r}_i \hat{r}_j - \frac{1}{3} \delta_{ij} \right) j_2(kr)
  \end{equation}
\end{formula}

\begin{proof}
  The integral must be a linear combination of $\delta_{ij}$ and
  $\hat{r}_i \hat{r}_j$ due to its tensor structure. Let the coefficients
  be $A$ and $B$,

  \begin{equation}
    \int\! d\Omega_k \, \hat{k}_i \hat{k}_j \exp^{i\bm{k}\cdot\bm{r}}
    = A \delta_{ij} + B \hat{r}_i \hat{r}_j.
  \end{equation}

   $\sum_{ij} \delta_{ij}$ on both sides gives,
  \begin{equation}
    \int \! d\Omega_k \exp^{i\bm{k}\cdot\bm{r}} = 4 \pi j_0(kr)
    = 3 A + B.
  \end{equation}

   Similarly, $\sum_{ij} \hat{r}_i \hat{r}_j$ on both sides gives,
  \begin{equation}
    \int \! d\Omega_k \cos^2 \theta \exp^{i\bm{k}\cdot\bm{r}}
      = A + B,
  \end{equation}
  where $\cos\theta = \hat{\bm{k}} \cdot \hat{\bm{r}}$.

  Formula~\ref{formula:cos2} gives,
  \begin{equation}
    A + B = \frac{4}{3}\pi \left[ j_0(kr) - 2 j_2(kr) \right]
  \end{equation}

  Solving two linear equations on $A$ and $B$ give,
  \begin{align}
    A &= \frac{1}{3} (j_0 + j_2)\\
    B &= - j_2
  \end{align}    
\end{proof}

%
% \Psi_ij
%
\begin{formula}
  \begin{equation}
    \Psi_{ij} (\bm{r}) \equiv \langle \Psi_i(\bm{p}) \Psi_j(\bm{q}) \rangle =
    \frac{1}{3} \Psi_0(r) \delta_{ij} - \left( \hat{r}_i \hat{r}_j - \frac{1}{3} \delta_{ij} \right) \Psi_2(r)
  \end{equation}
  where $\bm{r} \equiv \bm{p} - \bm{q}$ and,
  \begin{equation}
    \Psi_n(r) \equiv \frac{1}{2\pi^2} \int \! dk \, P(k) j_n(kr)
  \end{equation}
\end{formula}

\begin{proof}
  \begin{equation}
    \begin{split}
      \langle \Psi_i(\bm{p}) \Psi_j(\bm{q}) \rangle &=
      \int \! \frac{d^3 k}{(2\pi)^3} \,
        \hat{k}_i \hat{k}_j \frac{P(k)}{k^2} e^{i\bm{k}\cdot\bm{r}}\\
      &= \frac{1}{2\pi^2} \int\! dk \, P(k) \left[
         \frac{1}{3} j_0(kr) - \left( \hat{r}_i \hat{r}_j - \frac{1}{3}\delta_{ij} \right) j_2(kr) \right],
    \end{split}
  \end{equation}
  using Formula~\ref{formula-moment}.
\end{proof}

\begin{formula}
  \label{formula:inverse-bessel}
  The inverse of spherical Bessel transform
  \begin{equation}
    P(k) = k^2 \int 4\pi r^2 dr j_n(kr) \Psi_n(r).
  \end{equation}
\end{formula}

\begin{formula}
  \begin{equation}
    \Sigma(\bm{k}, \bm{r}) \equiv \mathrm{Var} \left[
      \bm{k} \cdot \bm{\Psi}(\bm{p}) - \bm{k} \cdot \bm{\Psi}(\bm{q}) \right]
    = 2\left[ k^2 \sigma_v^2 - k_i k_j \Psi_{ij}(\bm{r}) \right]
  \end{equation}

  \begin{equation}
    -\frac{1}{2} \Sigma(\bm{k}, \bm{r}) = 
      k^2 \left[ \frac{1}{3} \Psi_0(r) -
               \left( \cos\theta^2 - \frac{1}{3} \right) \Psi_2(r) - \sigma_v^2
          \right]
  \end{equation}
\end{formula}

The power spectrum of Zeldovich approximation:
\begin{equation}
  P_Z(k) = \int \! d^3 r \, e^{-i\bm{k}\cdot\bm{r}}
                          \left[ e^{-\frac{1}{2} \Sigma(\bm{k}, \bm{r})} - 1 \right]
\end{equation}


\begin{equation}
  k^2 \sigma_v^2 = \textrm{Var}\left[ \bm{k}\cdot\bm{\Psi} \right]
\end{equation}

\begin{equation}
  \sigma_v^2 = \frac{1}{6\pi^2}\int \! P(k) dk
\end{equation}

Linear limit

\begin{align}
  P_Z(k) &\approx - \int \! d^3 r \, e^{-i\bm{k}\cdot\bm{r}}
           \frac{1}{2}\Sigma(\bm{k}, \bm{q})\\
           &= k^2 \int\!d^3 r \, e^{-i\bm{k}\cdot\bm{r}}
               \left[ \frac{1}{3} \Psi_0(r) - \sigma_v^2
                 - \frac{2}{3} \mathcal{P}_2(\cos\theta) \Psi_2(r) \right]\\
               &= 4\pi k^2 \int\! r^2 d r \,
               \left[ \frac{1}{3} \Psi_0(r) j_0(kr)
                      + \frac{2}{3} \Psi_2(r) j_2(kr) \right],
\end{align}
where the constant term is zero,
\begin{equation}
  k^2 \sigma_v^2 \int \! d^3 r \, e^{-i\bm{k}\cdot\bm{r}} 
  =  k^2 \sigma_v^2 \delta_D(\bm{k}) = 0.
\end{equation}

The inverse formula (\ref{formula:inverse-bessel}) recovers the linear
power spectrum $P_Z(k) \approx P(k)$.

  

\label{LastPage}
\end{document}

%
% 
%
