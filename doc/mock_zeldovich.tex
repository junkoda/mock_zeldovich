\documentclass[a4paper,11pt]{article}
\usepackage{amsmath}
\usepackage{amssymb}
%\usepackage{fancyhdr}
\usepackage{hyperref}

% Use Times fonts
\usepackage{mathptmx}
\usepackage[scaled]{helvet}
\renewcommand{\ttdefault}{pcr}
\usepackage{bm}

\setlength{\oddsidemargin}{0pt}   %%% left margin
\setlength{\textwidth}{159.2mm}
\setlength{\topmargin}{0mm}
\setlength{\headheight}{10mm}
\setlength{\headsep}{10mm}         %%% length between header and test
\setlength{\textheight}{219.2mm}

\setlength{\parindent}{0pt}

%\pagestyle{fancy}
%\renewcommand{\headrulewidth}{0pt}
%\fancyhf{}
%\rhead{JK 0000 - \thepage}
%\rfoot{\url{http://www.github.com/junkoda/mock_zeldovich}}

\begin{document}

\section{Analytical Theory}

The Zeldovich approximation is the first-order solution to the
Lagrangian pertubation theory, which relates the unperturbed
Lagrangian coordinate $\bm{q}$ to the pertubed Eularian position
$\bm{x}$,
\begin{equation}
  \bm{x} = \bm{q} + \bm{\Psi}(\bm{q})
\end{equation}

The displacement in redshift space is only difference by a constant factor,
\begin{equation}
  \bm{s} = \bm{x} + \frac{1}{aH(a)} v_z \bm{e}_z
         = \bm{q} + \bm{\Psi}^s(\bm{q}),
\end{equation}
where we assume that the line of sight is in the $z$ direction.

\begin{align}
  \begin{cases}
  \Psi_i^s &= \Psi_i \mbox{  for $i=1,2$}\\
  \Psi_3^s &= \Psi_3
  \end{cases}m
\end{align}

The Zeldovich appromation to the displacement vector $\Psi$ is,
\begin{equation}
  \delta(\bm{x}) = - \bm{\nabla}_q \cdot \bm{\Psi}(\bm{q})
\end{equation}
where $\delta$ is the density contrast $\delta = n(\bm{x}/\langle n
\rangle - 1$ for some density field n; the angle braket denotes the
ensamble average. The vorticity is a decaying mode and often set to zero,
\begin{equation}
  \bm{\nabla}_q \times \bm{\Psi}(\bm{q}) = 0.
\end{equation}

\begin{equation}
  \delta(\bm{k}) = \int d^3 q \, e^{-i \bm{k} \cdot \bm{q}}
                                 e^{-i \bm{k} \cdot \bm{\Psi}(\bm{q})}
\end{equation}

The power spectrum, defined by,
\begin{equation}
  \left\langle \delta(\bm{k}) \delta(\bm{k}') \right\rangle
  = (2\pi)^3 \delta_D(\bm{k} - \bm{k}') P(\bm{k}),
\end{equation}
is
\begin{equation}
  P(\bm{k}) = \int d^3 r \,  e^{-i \bm{k} \cdot \bm{r}} \left \langle
  e^{-i \bm{k} \cdot \left[ \bm{\Psi}(\bm{p}) - \bm{\Psi}(\bm{q}) \right]} \right \rangle,
\end{equation}
where $\bm{r} \equiv \bm{p} - \bm{q}$.

The ensamble average can be computed with the charactersic function formula
for a random Gaussian variable $X$ with $\langle X \rangle = 0$,
\begin{equation}
  \left\langle e^{-t X} \right\rangle = e^{-\frac{1}{2} \sigma^2 t^2},
\end{equation}
where $\sigma^2 = \left\langle X^2 \right\rangle$.

\begin{equation}
  P(\bm{k}) = \int d^3 r \, e^{-i \bm{k} \cdot \bm{r}}
    e^{-\frac{1}{2} \Sigma(\bm{k}, \bm{r})},
\end{equation}
where $\Sigma(\bm{k}, \bm{r})$ is the variance of $\bm{k} \cdot \left[
  \bm{\Psi}(\bm{p}) - \bm{\Psi}(q) \right]$:

\begin{equation}
  \Sigma(\bm{k}, \bm{r}) = 2 \sum_i k_i^2 \left[ \sigma_i^2 - \xi_{ii}(\bm{r}) \right],
\end{equation}
where
\begin{align}
  \sigma^2_i       &= \left\langle \Psi_i^2 \right\rangle,\\
  \xi_{ij}(\bm{r}) &= \left\langle \Psi_i(\bm{p}) \Psi_j(\bm{q}) \right\rangle
                   = \delta_{ij} \left\langle \Psi_i(\bm{p}) \Psi_j(\bm{q}) \right\rangle,
\end{align}
and $\delta_{ij}$ is the Kronecker delta.

The Zeldovich displacement field in Fourier space is,
\begin{equation}
  \bm{\Psi}(\bm{k}) = \frac{i\bm{k}}{k^2} \delta(\bm{k}).
\end{equation}



\begin{equation}
  \xi_{ij}(\bm{r}) = \int \frac{d^3 k}{(2\pi)^3}\,
                    e^{i\bm{k} \cdot \bm{r}} \frac{k_i}{k} \frac{k_j}{k}
                    \frac{P(\bm{k})}{k^2}
\end{equation}

The off-diagonal terms are zero from the asymmetry across the axies
and the diagonal term can be computed with Legendre polynomials,

\begin{align}
  \mathcal{P}_0(\mu) = 1\\
  \mathcal{P}_2(\mu) = \frac{1}{2} \left( 3 \mu^2 - 1 \right)
\end{align}

\begin{equation}
  \int \mathcal{P}_m(\mu) \mathcal{P}_n(\mu) d\mu =
     \frac{2}{2 n + 1} \delta_{\ell n}
\end{equation}

\begin{equation}
  \xi_{22}(\bm{r}) = \frac{1}{3} \int \frac{d^3 k}{(2\pi)^3} \,
             \left[ 2 \mathcal{P}_2(\mu) + \mathcal{P}_0 \right] P(\bm{k})/k^2
             = \frac{2}{3}  \xi_2(k) - \frac{1}{3} \xi_2(0)
\end{equation}
where $\xi_\ell$ is the multipoles of the displacement correlation function

\section{Discrete Theory}


\label{LastPage}
\end{document}

%
% 
%
